\documentclass[11pt,a4paper]{article}
\usepackage[utf8]{inputenc}
\usepackage[T1]{fontenc}
\usepackage{blindtext}
\usepackage{enumitem}
\usepackage{hyperref}

\begin{document}


\section{Informações Estudante}
\begin{itemize}
\item Aluno: João Henrique Pereira de Almeida
\item Matrícula: 15/0132042
\item Perfil Github: \href{http://github.com/joao-henrique/}{http://github.com/joao-henrique/}
\item email: joaohenrique.p.almeida@gmail.com
\item Disciplina: Fundamentos de Sistemas Operacionais
\end{itemize}


\section{Informações Importantes}
\subsection{Repositório}
O codigo dessa atividade se encontra no repositório abaixo, sob licença MIT
\url{https://github.com/joao-henrique/FSO-2016.1/tree/master/trab01}
\subsection{Sistema Operacional}

Foi usado para o desenvolvimento das atividades propostas o ElementaryOS - Loki, sistema operacional baseado no Ubuntu 16.04

\subsection{Ambiente de Desenvolvimento }
- O código encontrado nos três exercicios funcionaram em
conformidade em ambiente GNU/Linux OS, juntamente com o GCC-5

\section{Instruções}
\subsection{Instruções para compilação}
Aqui será listado os comandos que devem ser executados no terminal, dentro da pasta raiz.

\subsubsection{Questão 01}
No diretório raiz do projeto utilize os seguintes comandos
\begin{verbatim}
$ cd q01/
$ make
\end{verbatim}


\subsection{Casos de Teste}

\subsubsection{Questão 01}
\begin{itemize}
  \item Na entrada dos vertices do triângulo  não é verificado se a cadeia de
    strings informada são letras ou numeros
  \begin{itemize}
    \item Exemplo 01: Entrada: FSODisciplina, Saída: "The triangle does not exists";
    \item Exemplo 02: Entrada:1,8,5,6,7,9 , \\
    Saída: " The area of the triangle is: 8.00; The side of the vertice n1 is: 4.47; \\
    The side of the vertice n2 is: 3.61;\\
    The side of the vertice n3 is: 6.08;\\
    The perimeter of the triangle is: 14.16;\\
  \end{itemize}
\item É verificado se os pontos dos vertices informados estão ma mesma reta do plano
  \begin{itemize}
    \item Exemplo: Entrada: 1,2,3,4,5,6, Saída: "The triangle does not exists";
  \end{itemize}
\end{itemize}

\subsection{Limitações conhecidas}
\subsubsection{Questão 01}
\begin{itemize}
  \item Não há um tratamento dos valores de entrada no sistema
\end{itemize}

\end{document}
