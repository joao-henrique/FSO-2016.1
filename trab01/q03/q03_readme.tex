\documentclass[11pt,a4paper]{article}
\usepackage[utf8]{inputenc}
\usepackage[T1]{fontenc}
\usepackage{blindtext}
\usepackage{enumitem}
\usepackage{hyperref}

\begin{document}


\section{Informações Estudante}
\begin{itemize}
\item Aluno: João Henrique Pereira de Almeida
\item Matrícula: 15/0132042
\item Perfil Github: \href{http://github.com/joao-henrique/}{http://github.com/joao-henrique/}
\item email: joaohenrique.p.almeida@gmail.com
\item Disciplina: Fundamentos de Sistemas Operacionais

\end{itemize}


\section{Informações Importantes}
\subsection{Repositório}
O codigo dessa atividade se encontra no repositório abaixo, sob licença MIT
\url{https://github.com/joao-henrique/FSO-2016.1/tree/master/trab01}
\subsection{Sistema Operacional}


Foi usado para o desenvolvimento das atividades propostas o ElementaryOS - Loki, sistema operacional baseado no Ubuntu 16.04

\subsection{Ambiente de Desenvolvimento }
- O código encontrado nos três exercicios funcionaram em
conformidade em ambiente GNU/Linux OS, juntamente com o GCC-5

\section{Instruções}


\subsection{Instruções para compilação}
Aqui será listado os comandos que devem ser executados no terminal, dentro da pasta raiz.

\subsubsection{Questão 03}
No diretório raiz do projeto utilize os seguintes comandos
\begin{verbatim}
$ cd q03/
$ make
\end{verbatim}

\subsection{Casos de Teste}

\subsubsection{Questão 03}

Entrada:
\begin{verbatim}
 $ Universidade de Brasilia
\end{verbatim}
Saída esperada:
\begin{verbatim}
 $ Value of dPtr: 7.300000
 $ Value of number2: 7.300000
 $ Pointer to number1: 0x7fff90d0d218
 $ Pointer to dPtr: 0x7fff90d0d218
 $ Put a string: Universidade de Brasilia
 $ Compare to s1 and s2: 0
 $ Concat of s1 and s2: UniversidadeUniversidade
 $ Len of |UniversidadeUniversidade| is |24|
\end{verbatim}

\subsection{Limitações conhecidas}
\subsubsection{Questão 03}
\begin{itemize}
\item
O valor impresso decorrente do enunciado que contempla o item anterior é igual ao
valor do endereço gravado em dPrt?
\begin{verbatim}
    Não, é apenas um ponteiro indicando um endereço
    \end{verbatim}

\item
  A execução do item anterior pode provocar algum erro em tempo de execução?
\begin{verbatim}
    Sim, pode ocorrer um problema com a concatenação e o tamanho não suficiente \\
    para armazenar ambas strings
\end{verbatim}



\end{itemize}

\end{document}
