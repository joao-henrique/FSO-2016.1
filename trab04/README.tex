\documentclass[11pt,a4paper]{article}
\usepackage[utf8]{inputenc}
\usepackage[T1]{fontenc}
\usepackage{blindtext}
\usepackage{enumitem}
\usepackage{hyperref}

\begin{document}

\section{Informações Estudante}
\begin{itemize}
\item Aluno: João Henrique Pereira de Almeida
\item Matrícula: 15/0132042
\item Perfil Github: \href{http://github.com/joao-henrique/}{http://github.com/joao-henrique/}
\item email: joaohenrique.p.almeida@gmail.com
\item Disciplina: Fundamentos de Sistemas Operacionais
\end{itemize}


\section{Informações Importantes}
\subsection{Repositório}
O código dessa atividade se encontra no repositório abaixo, sob licença MIT
\url{https://github.com/joao-henrique/FSO-2016.1/tree/master/trab04}
\subsection{Sistema Operacional}

Foi usado para o desenvolvimento das atividades propostas o ElementaryOS - Loki, sistema operacional baseado no Ubuntu 16.04

\subsection{Ambiente de Desenvolvimento }
- O código encontrado nos três exercicios funcionaram em
conformidade em ambiente GNU/Linux OS, juntamente com o GCC-5

\section{Instruções}
\subsection{Instruções para compilação}
Aqui será listado os comandos que devem ser executados no terminal, dentro da pasta raiz.

\subsubsection{Questão 02}
No diretório raiz do projeto utilize os seguintes comandos
\begin{verbatim}
$ make
\end{verbatim}
Será listado no terminal instruções que auxiliaram no uso.\\
Um exempo que pode ser testado:
\begin{verbatim}
$ ./buscador dir/show/ test 3
\end{verbatim}

\subsection{Limitações conhecidas}
\subsubsection{Questão 02}
\begin{itemize}
  \item Não há um tratamento dos caminhos das pastas selecionadas
\end{itemize}

\section{Questão discursiva}
Norma que pressupõe
que o sistema operacional suporta os mesmos
recursos de segurança básicos encontrados na maioria dos sistemas de arquivos
UNIX. É possível definir duas distinções independentes entre arquivos\\
 - Compartilhaveis vs. Não Compartilhaveis \\
 - Variável vs. Estática. \\

Em geral, os ficheiros que diferem em qualquer um destes aspectos devem ser
localizados em pastas diferentes. Isto torna mais fácil
para armazenar arquivos com diferentes características de uso em diferentes
sistemas de arquivos.

Arquivos compartilhaveis são aqueles que podem ser armazenadas utilizando
um hospedeiro. Por exemplo os arquivos no diretório
home do usuário são compartilháveis enquanto arquivos de bloqueio do dispositivo não são.
arquivos estático incluem binários, bibliotecas, arquivos de documentação e
outros arquivos que não mudam sem sistemaa intervenção do administrador.

Esta norma permite:\\
- Prever a localização de arquivos instalados e diretórios \\
- Definir a especificação dos princípios orientadores para cada área do sistema de arquivos\\
- Especificando os arquivos mínimos e diretórios necessários,\\
- Enumerando exceções aos princípios

\end{document}
