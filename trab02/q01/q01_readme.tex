\documentclass[11pt,a4paper]{article}
\usepackage[utf8]{inputenc}
\usepackage[T1]{fontenc}
\usepackage{blindtext}
\usepackage{enumitem}
\usepackage{hyperref}

\begin{document}


\section{Informações Estudante}
\begin{itemize}
\item Aluno: João Henrique Pereira de Almeida
\item Matrícula: 15/0132042
\item Perfil Github: \href{http://github.com/joao-henrique/}{http://github.com/joao-henrique/}
\item email: joaohenrique.p.almeida@gmail.com
\item Disciplina: Fundamentos de Sistemas Operacionais
\end{itemize}


\section{Informações Importantes}
\subsection{Repositório}
O codigo dessa atividade se encontra no repositório abaixo, sob licença MIT
\url{https://github.com/joao-henrique/FSO-2016.1/tree/master/trab02}
\subsection{Sistema Operacional}

Foi usado para o desenvolvimento das atividades propostas o ElementaryOS - Loki, sistema operacional baseado no Ubuntu 16.04

\subsection{Ambiente de Desenvolvimento }
- O código encontrado nos três exercicios funcionaram em
conformidade em ambiente GNU/Linux OS, juntamente com o GCC-5

\section{Instruções}
\subsection{Instruções para compilação}
Aqui será listado os comandos que devem ser executados no terminal, dentro da pasta raiz.

\subsubsection{Questão 01}
No diretório raiz do projeto utilize os seguintes comandos
\begin{verbatim}
$ cd trab02/
$ cd q01/
$ make
\end{verbatim}


\subsection{Casos de Teste}

\subsubsection{Questão 01}
\begin{itemize}
  \item O trecho de codigo desenvolvido não apresentou limitação aparente
\end{itemize}


\subsection{Limitações conhecidas}
\subsubsection{Questão 01}
\begin{itemize}
  \item Quando executado o comando make no diretório a seguinte mensagem é esperada:

  \begin{verbatim}
  $ Waiting for the SIGALRM from son!
   \end{verbatim}

  Depois da mensagem exibida esperamos 5 segundos para o final do códigos \\

  \begin{verbatim}
   $ Son fished
   $ The SIGALRM arived! 
  \end{verbatim}





\end{itemize}


\end{document}
